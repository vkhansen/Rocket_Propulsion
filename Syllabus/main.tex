\documentclass[12pt]{article}

\usepackage[utf8]{inputenc}
\usepackage{geometry}
\geometry{a4paper, margin=1in}
\usepackage{hyperref}
\hypersetup{
    colorlinks=true,
    linkcolor=blue,
    filecolor=magenta,      
    urlcolor=cyan,
}
\usepackage{graphicx}
\usepackage{doi}
\usepackage{fancyhdr}
\usepackage{xcolor}
\usepackage{booktabs}

\pagestyle{fancy}
\fancyhf{}
\lhead{Course Syllabus - Rocket Propulsion}
\rhead{Version 1.1}
\cfoot{\thepage}

\title{\textbf{Course Syllabus \\ Rocket Propulsion} \newline \small{Version 1.1, Revised: \today}}
\author{Lecturer: Viggo Hansen\\
Aerospace Engineering, International School of    
Engineering (ISE)}
\date{Academic Year: 2024 \\ Semester: Second}

\begin{document}

\maketitle

% Hyperlinked Table of Contents
{
\hypersetup{linkcolor=black}
\tableofcontents
}

\section*{Course Information}
\label{sec:course_info}
\begin{itemize}
    \item \textbf{Course No.}: 2145473
    \item \textbf{Credits}: 3 credits (3-0-6)
    \item \textbf{Program}: Aerospace Engineering (International Program)
    \item \textbf{Level}: Undergraduate
    \item \textbf{Prerequisites}: 2183-221 Thermodynamics, 2183-222 Fluid Mechanics
\end{itemize}

\section*{Course Description}
Fundamentals of rocket propulsion, covering classical chemical rocket propulsion for launch, orbital, and interplanetary flight. Topics include:
\begin{itemize}
    \item Flight mission and performance
    \item Rocket equations
    \item Nozzle theory and design
    \item Future trends in rocket propulsion
    \item Preliminary design of engine components
\end{itemize}

\section*{Course Objectives}
Upon completing this course, students will:
\begin{itemize}
    \item Classify rocket engine types and identify the roles of main components
    \item Analyze flight mission regimes and determine flight performance
    \item Perform conceptual and preliminary design of rocket engines
\end{itemize}

\section*{Course Outline}
\begin{tabular}{|l|p{14cm}|}
\hline
\textbf{Week} & \textbf{Topic} \\
\hline
1-2 & Introduction to Rocket Engines (Sutton Ch. 2) \\
3 & Nozzle Theory and Thermodynamic Relations (Sutton Ch. 3) \\
4 & Flight Mission and Performance (Sutton Ch. 4) \\
5 & Chemical Rocket Propellant Performance Analysis (Sutton Ch. 5) \\
6-10 & Liquid Propellant Rocket Propulsion (Sutton Ch. 6-10) \\
    & Individual Project or Exam \\
11-13 & Solid Propellant Rocket Propulsion (Sutton Ch. 12-15) \\
14-15 & Hybrid Propellant Rocket Propulsion Design (Sutton Ch. 16) (Group Project) \\
\hline
\end{tabular}

\section*{Evaluation}
\begin{itemize}
    \item Weekly Quizzes (Paper-Based, 1 Note Page Allowed) Topics to match Sutton chapters in Course Outline: 30\%
    \item Homework Assigned Weekly via MCV (Submit as source code, pdf, to GitHub Repo): 30\%
    \item Final Design Group Project: 40\%
\end{itemize}

\subsection*{Homework Notes:}
\begin{itemize}
    \item \textbf{Contributions}: Homework can be submitted as improvements to existing code in the course repository, including:
    \begin{itemize}
        \item Additional vectorized Python functions relevant to the coursework (optimizations are greatly encouraged)
        \item Jupyter Lab notebooks solving specific textbook problems or problems created by students or the instructor
        \item Corrections to any existing software or content
    \end{itemize}
    There is no requirement to use software tools; assignments can also be submitted as PDF or Word documents.
    \item \textbf{Submission}: Submit homework code and the final project via the Course GitHub Repository in MATLAB, Python, or other relevant code formats.
    \item \textbf{GitHub Access}: Email your GitHub username to \href{mailto:vkhansen@eng.chula.ac.th}{vkhansen@eng.chula.ac.th} to be added to the repository.

    \item \textbf{Google Groups Mailing List:} Google Groups for discussions, updates, and Q\&A:
        \begin{itemize}
            \item \href{https://groups.google.com/g/rocket-propulsion-2145473}{Admin Panel: rocket-propulsion-2145473}
            \item \href{mailto:rocket-propulsion-2145473@googlegroups.com}{Email: rocket-propulsion-2145473@googlegroups.com}
        \end{itemize}

    
\end{itemize}

\section*{Online Resources}
\begin{itemize}
    \item Course GitHub Repository (Python/MATLAB/Jupyter Notebook):\\
    \href{https://github.com/vkhansen/rocket_propulsion.git}{Course Repository}
    \item NotebookLM:\\
    \href{https://notebooklm.google.com/notebook/c7f10837-5c21-46ee-af09-273d38bb41a3}{Google NotebookLM}
    \item Jeerasak Pitakarnnop Materials (pdf password "aeroise"):\\
    \href{https://pitakarnnop.wordpress.com/engineering-courses/rocket-propulsion}{Web}
\end{itemize}

\section*{Reading List}
\textbf{Required Textbook:}\\
Sutton, G. P., and Biblarz, O., \textit{Rocket Propulsion Elements}, 9th ed., Wiley, 2017.

\begin{thebibliography}{9}
\bibitem{Hagemann1998}
Gerald Hagemann, Hans Immich, Thong Van Nguyen, and Gennady E. Dumnov.\\
\textit{Advanced Rocket Nozzles}.\\
Journal of Propulsion and Power, 14(5):620-629, 1998.

\bibitem{Hill1992}
Philip G. Hill and Carl R. Peterson.\\
\textit{Mechanics and Thermodynamics of Propulsion}.\\
2nd edition, Prentice Hall, 1992.\\
ISBN: 978-81-317-2951-9

\bibitem{Huzel1967}
Dieter K. Huzel and David H. Huang.\\
\textit{Design of Liquid Propellant Rocket Engines}.\\
National Aeronautics and Space Administration, Washington, D.C., 1967.\\
(NASA SP-125)
\end{thebibliography}

\section*{Document History}
\begin{itemize}
    \item Version 1.0 - 01/21/2024 - Initial draft by Viggo Hansen
    \item Version 1.1 - \today - Revised to include detailed submission methods for homework and projects via GitHub, aligning with Code-First Learning principles.
\end{itemize}

\end{document}