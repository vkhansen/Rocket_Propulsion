\documentclass[12pt]{article}

\usepackage{geometry}
\geometry{a4paper, margin=1in}
\usepackage{fancyhdr}
\usepackage{mdframed} % For drawing boxes (already in your document)

% Your existing packages
\usepackage{amsmath}
\usepackage{amssymb}
\usepackage{booktabs}
\usepackage{hyperref}
\usepackage{tabularx}

\hypersetup{
    colorlinks=true,
    linkcolor=blue,
    filecolor=magenta,      
    urlcolor=cyan,
}

% Define document control information
\newcommand{\documentversion}{Version 1.0}
\newcommand{\documentdate}{\today}

\title{\textbf{Course Syllabus - Rocket Propulsion} \\ \small{\documentversion, Revised: \documentdate}}
\author{Viggo K. Hansen \href{mailto:vkhansen@eng.chula.ac.th}{[Email]}}
\date{January 30, 2025}

\begin{document}

\maketitle

% Setup headers and footers
\pagestyle{fancy}
\fancyhf{}
\fancyhead[L]{\documentversion}
\fancyhead[R]{\documentdate}
\fancyfoot[C]{\thepage\ of \pageref{LastPage}}
\renewcommand{\headrulewidth}{0.4pt}
\renewcommand{\footrulewidth}{0.4pt}

\tableofcontents

% Content sections remain unchanged
\section{\hyperref[sec:course_info]{Course Information}}
\label{sec:course_info}
\begin{itemize}
    \item \textbf{Course No.}: 2145473
    \item \textbf{Credits}: 3 credits (3-0-6)
    \item \textbf{Program}: Aerospace Engineering (International Program)
    \item \textbf{Level}: Undergraduate
    \item \textbf{Prerequisites}: 2183-221 Thermodynamics, 2183-222 Fluid Mechanics
\end{itemize}

\section{\hyperref[sec:course_desc]{Course Description}}
\label{sec:course_desc}
Fundamentals of rocket propulsion, covering classical chemical rocket propulsion for launch, orbital, and interplanetary flight. Topics include:
\begin{itemize}
    \item Flight mission and performance
    \item Rocket equations
    \item Nozzle theory and design
    \item Future trends in rocket propulsion
    \item Preliminary design of engine components
\end{itemize}

\section{\hyperref[sec:objectives]{Course Objectives}}
\label{sec:objectives}
Upon completing this course, students will:
\begin{itemize}
    \item Classify rocket engine types and identify the roles of main components
    \item Analyze flight mission regimes and determine flight performance
    \item Perform conceptual and preliminary design of rocket engines
\end{itemize}

\section{\hyperref[sec:outline]{Course Outline}}
\label{sec:outline}
\begin{tabular}{lll}
\toprule
\textbf{Week} & \textbf{Topic} & \textbf{Sutton Chapter} \\
\midrule
1-2 & Introduction to Rocket Engines & Ch. 2 \\
3 & Nozzle Theory and Thermodynamic Relations & Ch. 3 \\
4 & Flight Mission and Performance & Ch. 4 \\
5 & Chemical Rocket Propellant Performance Analysis & Ch. 5 \\
6-10 & Liquid Propellant Rocket Propulsion & Ch. 6-10 \\
    & Individual Project & \\
11-13 & Solid Propellant Rocket Propulsion & Ch. 12-15 \\
14-15 & Hybrid Propellant Rocket Propulsion Design (Group Project) & Ch. 16 \\
\bottomrule
\end{tabular}

\section{\hyperref[sec:evaluation]{Evaluation}}
\label{sec:evaluation}
\begin{itemize}
    \item Weekly Quizzes: 30\% (Paper-based, one note page allowed)
    \item Homework: 30\% (Weekly assignments and individual projects submit via MCV or additional optional methods discussed below)
    \item Final Design Group Project: 40\%
\end{itemize}

\section{\hyperref[sec:software_development]{Propulsion System Engineering Code Development}}
\label{sec:software_development}
The course aims for foster modular and composable systems engineering skills, vital for the aerospace industry. Students may:
\begin{itemize}
    \item \textbf{Contribute to Code:} Improve existing code in the course repository:
          \begin{itemize}
              \item Add vectorized Python functions (optimizations encouraged)
              \item Create Jupyter Lab notebooks for textbook or custom problems
              \item Correct existing software or content
          \end{itemize}
    \item \textbf{How to contribute:}
          \begin{enumerate}
              \item \textbf{GitHub Submission}: Submit code (MATLAB, Python, etc.) to the course repository. \href{mailto:vkhansen@eng.chula.ac.th}{[Email]} GitHub username to instructor for access.
          \end{enumerate}
\end{itemize}

\section{\hyperref[sec:resources]{Online Resources}}
\label{sec:resources}
\begin{itemize}
    \item Course GitHub Repository: \href{https://github.com/vkhansen/rocket_propulsion.git}{Course Repository}
    \item Course NotebookLM (Provide gmail account email address for access): \href{https://notebooklm.google.com/notebook/c7f10837-5c21-46ee-af09-273d38bb41a3}{Google NotebookLM}
    \item Dr. Jeerasak Pitakarnnop Materials (pdf password "aeroise"): \href{https://pitakarnnop.wordpress.com/engineering-courses/rocket-propulsion}{Web}
\end{itemize}

\section{\hyperref[sec:google_group]{Google Group Details}}
\label{sec:google_group}
The Google Group for this course, rocket-propulsion-2145473, serves as the primary communication platform allowing asynchronous email communication.

How to Join:
\begin{enumerate}
    \item \textbf{Via Web}: Navigate to the group's page at \href{https://groups.google.com/g/rocket-propulsion-2145473}{Google Groups} and click on "Join group".
    \item \textbf{Via Email}: Send an email to \href{mailto:rocket-propulsion-2145473+subscribe@googlegroups.com}{rocket-propulsion-2145473+subscribe@googlegroups.com} with a subject or body indicating your request to join the group. 
\end{enumerate}

After joining, you can either:
\begin{itemize}
    \item \textbf{Post messages}: Directly email to \href{mailto:rocket-propulsion-2145473@googlegroups.com}{rocket-propulsion-2145473@googlegroups.com}, or 
    \item \textbf{Manage settings}: Access the group's page to adjust notification preferences, view archives, etc.
\end{itemize}

\textbf{Note on Digest Mode}: To manage your email load, consider enabling digest mode. This can be done through the group settings page by selecting the option to receive emails in digest form, where messages are batched together and sent in one email per day or week.

Participation in the Google Group is encouraged by sharing knowledge, asking questions, and staying updated with course developments.

\section{\hyperref[sec:reading]{Reading List}}
\label{sec:reading}
\textbf{Required Textbook:}\\
\vspace{1pt}

Sutton, G. P., and Biblarz, O., \textit{Rocket Propulsion Elements}, 9th ed., Wiley, 2017.

\begin{thebibliography}{9}
\bibitem{Hagemann1998}
Gerald Hagemann, Hans Immich, Thong Van Nguyen, and Gennady E. Dumnov. \textit{Advanced Rocket Nozzles}. Journal of Propulsion and Power, 14(5):620-629, 1998.

\bibitem{Hill1992}
Philip G. Hill and Carl R. Peterson. \textit{Mechanics and Thermodynamics of Propulsion}. 2nd edition, Prentice Hall, 1992. ISBN: 978-81-317-2951-9

\bibitem{Huzel1967}
Dieter K. Huzel and David H. Huang. \textit{Design of Liquid Propellant Rocket Engines}. National Aeronautics and Space Administration, Washington, D.C., 1967. (NASA SP-125)
\end{thebibliography}

\label{LastPage}
\end{document}