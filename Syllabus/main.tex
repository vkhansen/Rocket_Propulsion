\documentclass[12pt]{article}

\usepackage{amsmath}
\usepackage{amssymb}
\usepackage{booktabs}
\usepackage{hyperref}
\usepackage{tabularx}
\usepackage{mdframed} % For drawing boxes

\hypersetup{
    colorlinks=true,
    linkcolor=blue,
    filecolor=magenta,      
    urlcolor=cyan,
}

\title{Course Syllabus - Rocket Propulsion}
\author{Viggo Hansen}
\date{January 30, 2025}

\begin{document}

\maketitle

\tableofcontents

\section{\hyperref[sec:course_info]{Course Information}}
\label{sec:course_info}
\begin{itemize}
    \item \textbf{Course No.}: 2145473
    \item \textbf{Credits}: 3 credits (3-0-6)
    \item \textbf{Program}: Aerospace Engineering (International Program)
    \item \textbf{Level}: Undergraduate
    \item \textbf{Prerequisites}: 2183-221 Thermodynamics, 2183-222 Fluid Mechanics
\end{itemize}

\section{\hyperref[sec:course_desc]{Course Description}}
\label{sec:course_desc}
Fundamentals of rocket propulsion, covering classical chemical rocket propulsion for launch, orbital, and interplanetary flight. Topics include:
\begin{itemize}
    \item Flight mission and performance
    \item Rocket equations
    \item Nozzle theory and design
    \item Future trends in rocket propulsion
    \item Preliminary design of engine components
\end{itemize}

\section{\hyperref[sec:objectives]{Course Objectives}}
\label{sec:objectives}
Upon completing this course, students will:
\begin{itemize}
    \item Classify rocket engine types and identify the roles of main components
    \item Analyze flight mission regimes and determine flight performance
    \item Perform conceptual and preliminary design of rocket engines
\end{itemize}

\section{\hyperref[sec:outline]{Course Outline}}
\label{sec:outline}
\begin{tabular}{lll}
\toprule
\textbf{Week} & \textbf{Topic} & \textbf{Sutton Chapter} \\
\midrule
1-2 & Introduction to Rocket Engines & Ch. 2 \\
3 & Nozzle Theory and Thermodynamic Relations & Ch. 3 \\
4 & Flight Mission and Performance & Ch. 4 \\
5 & Chemical Rocket Propellant Performance Analysis & Ch. 5 \\
6-10 & Liquid Propellant Rocket Propulsion & Ch. 6-10 \\
    & Individual Project or Exam & \\
11-13 & Solid Propellant Rocket Propulsion & Ch. 12-15 \\
14-15 & Hybrid Propellant Rocket Propulsion Design (Group Project) & Ch. 16 \\
\bottomrule
\end{tabular}

\section{\hyperref[sec:evaluation]{Evaluation}}
\label{sec:evaluation}
\begin{itemize}
    \item Weekly Quizzes (Paper-Based, 1 Note Page Allowed) Topics to match Sutton chapters in Course Outline: 30\%
    \item Homework Assigned Weekly via MCV (Submit as source code or pdf, to GitHub Repo, Via Email, MCV or hardcopy in class): 30\%
    \item Final Design Group Project: 40\%
\end{itemize}

\subsection{\hyperref[sec:homework]{Homework VCS Systems:}}
\label{sec:homework}
The software codes cooperatively developed in this course are designed to equip students with the essential skills of modular and composable systems engineering—a critical and highly sought-after competency in the aerospace industry, particularly among space startups and leading organizations. By engaging with these codes, students will learn to construct complex systems by integrating modular components, fostering adaptability and efficiency in design. This hands-on approach not only deepens their understanding of rocket propulsion concepts but also prepares them to meet the dynamic demands of modern aerospace engineering, where the ability to rapidly assemble and reconfigure systems is paramount.
\begin{itemize}
    \item \textbf{Contributions}: Homework can be submitted as improvements to existing code in the course repository, including:
    \begin{itemize}
        \item Additional vectorized Python functions relevant to the coursework (optimizations are greatly encouraged)
        \item Jupyter Lab notebooks solving specific textbook problems or problems created by students or the instructor
        \item Corrections to any existing software or content
    \end{itemize}
    \textbf{Note:} There is no requirement to use software tools; assignments can be submitted as PDF, MS Word, or hardcopy documents.
    \item \textbf{Submission}: Ideally submit homework code and the final project via the Course GitHub repository in MATLAB, Python, or other relevant code formats. If preferred, students may send written homework notes via MVC, email or hard-copy in class.
    \item \textbf{GitHub Access}: Email your GitHub username to \href{mailto:vkhansen@eng.chula.ac.th}{vkhansen@eng.chula.ac.th} to be added to the repository.
    \item \textbf{Google Groups Mailing List:} Google Groups for discussions, updates, and Q\&A:
        \begin{itemize}
            \item \href{https://groups.google.com/g/rocket-propulsion-2145473}{Admin Panel: rocket-propulsion-2145473}
            \item \href{mailto:rocket-propulsion-2145473@googlegroups.com}{Email: rocket-propulsion-2145473@googlegroups.com}
        \end{itemize}
\end{itemize}

\section{\hyperref[sec:resources]{Online Resources}}
\label{sec:resources}
\begin{itemize}
    \item Course GitHub Repository (Provide Github username for contributor access): \href{https://github.com/vkhansen/rocket_propulsion.git}{Course Repository}
    \item Course NotebookLM (Provide gmail account email address for access): \href{https://notebooklm.google.com/notebook/c7f10837-5c21-46ee-af09-273d38bb41a3}{Google NotebookLM}
    \item Jeerasak Pitakarnnop Materials (pdf password "aeroise"): \href{https://pitakarnnop.wordpress.com/engineering-courses/rocket-propulsion}{Web}
\end{itemize}

\section{\hyperref[sec:reading]{Reading List}}
\label{sec:reading}
\textbf{Required Textbook:}\\
\vspace{1pt}

Sutton, G. P., and Biblarz, O., \textit{Rocket Propulsion Elements}, 9th ed., Wiley, 2017.

\begin{thebibliography}{9}
\bibitem{Hagemann1998}
Gerald Hagemann, Hans Immich, Thong Van Nguyen, and Gennady E. Dumnov. \textit{Advanced Rocket Nozzles}. Journal of Propulsion and Power, 14(5):620-629, 1998.

\bibitem{Hill1992}
Philip G. Hill and Carl R. Peterson. \textit{Mechanics and Thermodynamics of Propulsion}. 2nd edition, Prentice Hall, 1992. ISBN: 978-81-317-2951-9

\bibitem{Huzel1967}
Dieter K. Huzel and David H. Huang. \textit{Design of Liquid Propellant Rocket Engines}. National Aeronautics and Space Administration, Washington, D.C., 1967. (NASA SP-125)
\end{thebibliography}

\section{\hyperref[sec:history]{Document History}}
\label{sec:history}
\begin{itemize}
    \item Version 1.0 - 01/21/2024 - Initial draft by Viggo Hansen
    \item Version 1.1 - January 30, 2025 - Revised to include detailed submission methods for homework and projects via GitHub, aligning with Code-First Learning principles.
\end{itemize}

\end{document}