\documentclass{article}
\usepackage{amsmath}
\usepackage{geometry}
\geometry{margin=1in}

\begin{document}

\section{Explanation of Mistakes in Student Work for Problem 7 Part (b)}

The student's work for Problem 7 part (b) from the text "Rocket Propulsion Elements" contains several errors which led to incorrect results. Below is a detailed examination of these mistakes:

\subsection{Incorrect Calculation of Exit Velocity}

The student mistakenly used the specific impulse \( I_s \) directly as the exit velocity \( V_e \):

\begin{equation}
V_e = 237.1 \, \text{ft/s}
\end{equation}

This is incorrect because the specific impulse \( I_s \) is not directly equivalent to the exit velocity. The exit velocity should be calculated using the formula:

\begin{equation}
V_e = I_s \times g_0
\end{equation}

Where \( g_0 \) is the standard gravitational acceleration at sea level, \( 32.174 \, \text{ft/s}^2 \). Thus, the correct exit velocity should have been:

\begin{equation}
V_e = 237.1 \times 32.174 = 7622.4 \, \text{ft/s}
\end{equation}

\subsection{Unit Confusion}

There is clear confusion regarding the units in the student's work. The student wrote:

\begin{equation}
F = \dot{m} V_e + (p_e - p_a) A_e
\end{equation}

However, in the calculation, they did not convert the area \( A_e \) from square inches to square feet, which is necessary for dimensional consistency when dealing with pressures in psi:

\begin{equation}
A_e = \frac{1642}{144} \, \text{ft}^2
\end{equation}

The student's work lacks this conversion, leading to incorrect dimensional analysis and calculation errors.

\subsection{Incorrect Formula Application}

The student failed to correctly apply the thrust equation at optimum expansion. The correct formula as provided in the solutions manual is:

\begin{equation}
F_{\text{opt}} = F_t - (\dot{m} V_e + (p_e - p_a) A_e)
\end{equation}

However, the student's approach was flawed as they calculated:

\begin{equation}
\dot{m} V_e = 207,000 - (8.666 \times 14.696)
\end{equation}

This calculation is incorrect because it does not account for the proper dimensional conversion and misunderstands the application of the momentum thrust term \( \dot{m} V_e \). The correct approach should have been:

\begin{equation}
F_{\text{opt}} = F_t - ((p_e - p_a) \times A_e \times 144)
\end{equation}

Where the area conversion factor (144) converts \( A_e \) back to square inches for the pressure term.

\subsection{Impact on Specific Impulse Calculation}

Due to the incorrect thrust calculation, the student's specific impulse at optimum expansion \( I_{s,\text{opt}} \) was also incorrect:

\begin{equation}
I_{s,\text{opt}} = \frac{F_{\text{opt}}}{\dot{m}}
\end{equation}

The student calculated:

\begin{equation}
I_{s,\text{opt}} = \frac{316511.86}{873} = 361.56 \, \text{s}
\end{equation}

This result is incorrect because \( F_{\text{opt}} \) was miscalculated. The correct calculation should have been:

\begin{equation}
I_{s,\text{opt}} = \frac{216,900}{873 \times 32.174} = 248.5 \, \text{s}
\end{equation}

This value matches the solution provided in the manual.

\section{Conclusion}

The primary errors in the student's work were:
\begin{itemize}
    \item Misunderstanding and misapplication of the exit velocity calculation.
    \item Confusion with unit conversions, particularly between square inches and square feet.
    \item Incorrect application of the thrust formula at optimum expansion, leading to incorrect dimensional handling.
\end{itemize}

Correcting these errors would align the student's work with the solutions manual's calculations, providing accurate results for thrust and specific impulse at optimum expansion.

\end{document}