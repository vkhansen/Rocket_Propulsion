\documentclass{article}
\usepackage{amsmath}

\title{Rocket Propulsion Problems Solution - Sutton Ch 2}
\author{Viggo K. Hansen}
\date{January 27, 2025}

\begin{document}

\maketitle

\section{Problem 1: Rocket Propulsion Problem Solution}

\subsection{Problem Statement}
The following data are given for a certain rocket unit: thrust, 8896 N; propellant consumption, 3.867 kg/sec; velocity of vehicle, 400 m/sec; energy content of propellant, 6.911 MJ/kg. Assume 100\% combustion efficiency.

Determine:
\begin{itemize}
    \item[(a)] the effective velocity;
    \item[(b)] the kinetic jet energy rate per unit flow of propellant;
    \item[(c)] the internal efficiency;
    \item[(d)] the propulsive efficiency;
    \item[(e)] the overall efficiency;
    \item[(f)] the specific impulse;
    \item[(g)] the specific propellant consumption.
\end{itemize}

\subsection{Solution}

Given:
\begin{itemize}
    \item $F = 8896 \, \text{N}$
    \item $\dot{m} = 3.867 \, \text{kg/sec}$
    \item $u = 400 \, \text{m/sec}$
    \item $Q_R = 6.911 \, \text{MJ/kg}$
    \item $\eta_{\text{comb}} = 1.0$
\end{itemize}

\subsubsection{(a) Effective Velocity}
The effective velocity, $c$, is given by:
\[
c = \frac{F}{\dot{m}} = \frac{8896 \, \text{N}}{3.867 \, \text{kg/sec}} = 2300 \, \text{m/sec}
\]

\subsubsection{(b) Kinetic Jet Energy Rate per Unit Flow of Propellant}
The kinetic jet energy rate per unit flow of propellant, $E_k$, is:
\[
E_k = \frac{c^2}{2} = \frac{(2300 \, \text{m/sec})^2}{2} = 2.645 \times 10^6 \, \text{m}^2/\text{sec}^2 = 2.645 \, \text{MJ/kg}
\]

\subsubsection{(c) Internal Efficiency}
The internal efficiency, $\eta_i$, is:
\[
\eta_i = \frac{0.5mc^2}{\dot{m} Q_R \eta_{\text{comb}}} = \frac{0.5 \times 3.867 \times 2300^2}{3.867 \times 6.911 \times 10^6} = 0.383 = 38.3\%
\]

\subsubsection{(d) Propulsive Efficiency}
The propulsive efficiency, $\eta_p$, is:
\[
\eta_p = \frac{2u/c}{1 + (u/c)^2} = \frac{2 \times 400/2300}{1 + (400/2300)^2} = 0.3376 = 33.76\%
\]

\subsubsection{(e) Overall Efficiency}
The overall efficiency, $\eta$, is:
\[
\eta = \eta_i \eta_p = 0.383 \times 0.3376 = 0.1293 = 12.93\%
\]

Alternatively, it can be calculated as:
\[
\eta = \frac{Fu}{\dot{m} Q_R} = \frac{8896 \times 400}{3.867 \times 6.911 \times 10^6} = 0.133 = 13.3\%
\]

\subsubsection{(f) Specific Impulse}
The specific impulse, $I_s$, is:
\[
I_s = \frac{F}{\dot{m} g_0} = \frac{8896}{3.867 \times 9.81} = 234.5 \, \text{sec}
\]

\subsubsection{(g) Specific Propellant Consumption}
The specific propellant consumption, SFC, is:
\[
\text{SFC} = \frac{1}{I_s} = \frac{1}{234.5} = 0.00426 \, \text{sec}^{-1}
\]

\subsection{Conclusion}
The answers are:
\begin{itemize}
    \item[(a)] $2300 \, \text{m/sec}$
    \item[(b)] $2.645 \, \text{MJ/kg}$
    \item[(c)] $38.3\%$
    \item[(d)] $33.76\%$
    \item[(e)] $13.3\%$
    \item[(f)] $234.5 \, \text{sec}$
    \item[(g)] $0.00426 \, \text{sec}^{-1}$
\end{itemize}

\section{Problem 2: Rocket Propulsion Problem 4 Solution}

\subsection{Problem Statement}
For the rocket in Problem 1, calculate the specific power, assuming a propulsion system dry mass of 80 kg and a duration of 3 min.

\subsection{Given Data from Problem 1}
\begin{itemize}
    \item Thrust, \( F = 8896 \, \text{N} \)
    \item Propellant consumption rate, \( \dot{m} = 3.867 \, \text{kg/sec} \)
    \item Effective velocity, \( c = 2300 \, \text{m/sec} \)
\end{itemize}

\subsection{Additional Data for Problem 2}
\begin{itemize}
    \item Dry mass of propulsion system, \( m_f = 80 \, \text{kg} \)
    \item Duration, \( t = 3 \, \text{min} = 180 \, \text{sec} \)
\end{itemize}

\subsection{Solution}

\subsubsection{Specific Power Calculation}
Specific power is given by:
\[
\text{Specific Power} = \frac{P_{\text{jet}}}{m_0}
\]
where \( P_{\text{jet}} = 0.5Fc \) and \( m_0 \) is the total initial mass of the rocket including propellant.

First, calculate the total initial mass \( m_0 \):
\[
m_0 = \dot{m} \times t + m_f = (3.867 \, \text{kg/sec}) \times (180 \, \text{sec}) + 80 \, \text{kg} = 776.06 \, \text{kg}
\]

Next, calculate the jet power \( P_{\text{jet}} \):
\[
P_{\text{jet}} = 0.5 \times F \times c = 0.5 \times 8896 \, \text{N} \times 2300 \, \text{m/sec} = 10,230,400 \, \text{W}
\]

Finally, calculate the specific power:
\[
\text{Specific Power} = \frac{P_{\text{jet}}}{m_0} = \frac{10,230,400 \, \text{W}}{776.06 \, \text{kg}} = 13.18 \, \text{kW/kg}
\]

\subsection{Conclusion}
The specific power for the rocket is:
\[
\boxed{13.18 \, \text{kW/kg}}
\]

\section{Problem 3: Rocket Propulsion Problem 7 Solution}

\subsection{Problem Statement}
For a solid propellant rocket motor with a sea-level thrust of 207,000 lbf, determine:
\begin{itemize}
    \item[(a)] the (constant) propellant mass flow rate $\dot{m}$ and the specific impulse $I_s$ at sea level;
    \item[(b)] the altitude for optimum nozzle expansion as well as the thrust and specific impulse at this optimum condition and (c) at vacuum conditions.
\end{itemize}

Given:
\begin{itemize}
    \item Initial total mass of the rocket motor is 50,000 lbm and its propellant mass fraction is 0.90.
    \item The residual propellant (called slivers) combustion stops when the chamber pressure falls below a deflagration limit, which amounts to 3\% of the burnt.
    \item Burn time is 50 seconds.
    \item Nozzle throat area ($A_t$) is 164.2 in$^2$.
    \item Area ratio ($A_e/A_t$) is 10.
    \item Chamber pressure ($p_c$) is 780 psia and the pressure ratio ($p_e/p_c$) across the nozzle may be taken as 90.0.
\end{itemize}

\subsection{Solution}

\subsubsection{(a) Propellant Mass Flow Rate and Specific Impulse at Sea Level}

First, calculate the usable propellant mass:
\[
m_p = m_0 \kappa_f = 50,000 \times 0.90 = 45,000 \, \text{lbm}
\]
Since 3\% of the burnt propellant remains as residual:
\[
\text{usable part} = 45,000 \times 0.97 = 43,650 \, \text{lbm}
\]

Now, calculate the propellant mass flow rate $\dot{m}$:
\[
\dot{m} = \frac{m_p}{t} = \frac{43650}{50} = 873.0 \, \text{lbm/sec}
\]

Calculate the total impulse $I_t$:
\[
I_t = F \times t = 207,000 \times 50 = 10,350,000 \, \text{lb-sec}
\]

Finally, calculate the specific impulse $I_s$ at sea level:
\[
I_s = \frac{I_t}{W} = \frac{10,350,000}{43650} = 237.1 \, \text{sec}
\]

\subsubsection{(b) Altitude for Optimum Nozzle Expansion}

From Sutton, 9th Edition, Chapter 3, Fig. 3-4, if $k = 1.25$ and with the given chamber pressure (780 psia) and area ratio (10), at optimum expansion, $p_2 = p_3 = \frac{780}{90} = 8.666 \, \text{psia}$.

The nozzle exit area $A_e$ is:
\[
A_e = 10 A_t = 10 \times 164.2 = 1642 \, \text{in}^2
\]

Using Eq. 2-13 from Sutton to solve for the momentum thrust at optimum expansion:
\[
\dot{m} v_e = 207,000 - (8.666 \times 14.696) = 216,900 \, \text{lbf}
\]

Calculate the specific impulse at optimum expansion:
\[
I_s = \frac{216,900}{873} = 248.45 \, \text{sec}
\]

From Appendix 2 in Sutton, the altitude for 8.666 psia is approximately 4,200 meters.

\subsubsection{(c) Vacuum Conditions}

For vacuum conditions, $p_3 = 0$. Using Eq. 2-14:
\[
F = 216,900 + 8.67 \times 1642 = 231,000 \, \text{lbf}
\]

Calculate the specific impulse in vacuum:
\[
I_s = \frac{231,000}{873} = 264.6 \, \text{sec}
\]

\subsection{Conclusion}
The results are:
\begin{itemize}
    \item[(a)] $\dot{m} = 873.0 \, \text{lbm/sec}$, $I_s = 237.1 \, \text{sec}$
    \item[(b)] Altitude for optimum nozzle expansion: 4,200 meters, Thrust: 216,900 lbf, $I_s = 248.45 \, \text{sec}$
    \item[(c)] Thrust in vacuum: 231,000 lbf, $I_s = 264.6 \, \text{sec}$
\end{itemize}

\end{document}